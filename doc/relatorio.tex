\documentclass[a4paper]{article}

\usepackage{graphicx}

\usepackage{hyperref}
\hypersetup{
    colorlinks=true,
    linkcolor=black,      
    urlcolor=blue,
}
\begin{document}
\begin{titlepage}
	\newcommand{\HRule}{\rule{\linewidth}{0.5mm}}
	\center
	\textsc{\LARGE INSTITUTO SUPERIOR TÉCNICO}\\[1.0cm]
	\textsc{\Large Sistemas Distribuídos}\\[0.5cm]
	\textsc{\large 3º Ano, 2º Semestre 2018/2019}\\[0.2cm]
	
	\HRule\\[0.4cm]
	{\huge\bfseries Relatório - Segunda Parte}\\[0.2cm]
	\HRule\\[1.5cm]
	
	\begin{minipage}{0.4\textwidth}
		\begin{flushleft}
		\large
		\textit{Autores}\\
		Filipe Marques\\
		Jorge Martins\\
		Paulo Dias
		\end{flushleft}
	\end{minipage}
	~
	\begin{minipage}{0.4\textwidth}
		\begin{flushright}
		\large
		\textit{Docente}\\
		Tomás Grelha da Cunha
		\end{flushright}
	\end{minipage}
	
	\vfill
	\large\href{https://github.com/tecnico-distsys/A41-ForkExec}{\textbf{A41-ForkExec}}
	\vfill
	{\large\today}
	\vfill
\end{titlepage}

\begin{titlepage}

\tableofcontents
\end{titlepage}

\section{Definição do Modelo de Faltas}
Assume-se que:
\begin{itemize}
\item O sistema é assíncrono e a comunicação pode omitir mensagens
\begin{itemize}
\item Apesar do projeto usar HTTP como transporte, deve assumir-se que outros protocolos  de menor fiabilidade podem ser usados
\end{itemize}
\item Existem \textbf{N} gestores de réplicas e \textbf{N} é constante e igual a 3
\item Os gestores de réplicas podem falhar silenciosamente mas não arbitrariamente
\item No máximo, existe uma minoria de gestores de réplica em falha em simultâneo
\end{itemize}
\section{Solução de Tolerância a Faltas}
\begin{figure}[h!]
	\includegraphics[scale=.3]{../../domainsimple.jpeg}
	\caption{Simplificação do domínio da solução implementada}
	\label{fig:domain}
\end{figure}
\section{Descrição e breve explicação da solução}

Para manter a interface para o \textit{Hub} a	\textit{FrondEnd} do \textit{points} continua a implementar as funções necessárias para retornar os valores para o \textit{Hub}, no entanto esta classe implementa o protocolo de replicação ativa.

Para o \textit{FrontEnd} o \textit{PointsClient} funciona apenas como ponto de ligação com o gestor de réplica.
\section{Descrição de otimizações/simplificações}
\section{Detalhe do protocolo (troca de mensagens)}
\subsection{Funções no Gestor de Réplica}
\subsubsection{\textit{read(String userEmail)}}
\begin{itemize}
\item Ao receber \textit{read(userEmail)}:
\begin{itemize}
\item[1.] Vai buscar a \textit{tag} e os \textit{pontos} associados ao \textit{userEmail}
\begin{itemize}
\item[1.1.] Se \textit{userEmail} não existe no sistema, então adiciona.
\end{itemize}
\item[2.] responde com \textbf{$Value=< pontos, tag >$} associado ao utilizador, em que: $tag=< seq, cid>$
\end{itemize}
\end{itemize}
\subsubsection{\textit{write(String userEmail, int points, Tag t)}}
\begin{itemize}
\item Ao receber \textit{write(userEmail,points, t)}:
\begin{itemize}
\item[1.] Vai buscar a \textit{tag} associada ao \textit{userEmail}
\begin{itemize}
\item[1.1.] Se \textit{userEmail} não existe no sistema, então adiciona.
\end{itemize}
\item[2.] Se \textit{t.getSeq() $>$ tag.getSeq()}:
\begin{itemize}
\item[2.1.] atualiza os \textit{pontos} do utilizador com \textit{points}
\item[2.2.] atualiza a \textit{tag} do utilizador com \textit{t}
\item[2.3.] responde \textit{ack}
\end{itemize}
\item[3.] Senão responde \textit{nack}
\end{itemize}
\end{itemize}
\subsection{Funções no \textit{Front End}}
\subsubsection{\textit{activateUser(String userEmail)}}
\subsubsection{\textit{pointsBalance(String userEmail)}}
\subsubsection{\textit{addPoints(String userEmail, int pointsToAdd)}}
\subsubsection{\textit{spendPoints(String userEmail, int pointsToSpend)}}
\end{document}
