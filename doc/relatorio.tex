\documentclass[a4paper]{article}

\usepackage{graphicx}

\usepackage{hyperref}
\hypersetup{
    colorlinks=true,
    linkcolor=black,      
    urlcolor=blue,
}
\begin{document}
\begin{titlepage}
	\newcommand{\HRule}{\rule{\linewidth}{0.5mm}}
	\center
	\textsc{\LARGE INSTITUTO SUPERIOR TÉCNICO}\\[1.0cm]
	\textsc{\Large Sistemas Distribuídos}\\[0.5cm]
	\textsc{\large 3º Ano, 2º Semestre 2018/2019}\\[0.2cm]
	
	\HRule\\[0.4cm]
	{\huge\bfseries Relatório - Segunda Parte}\\[0.2cm]
	\HRule\\[1.5cm]
	
	\begin{minipage}{0.4\textwidth}
		\begin{flushleft}
		\large
		\textit{Autores}\\
		86411 - Filipe Marques\\
		86456 - Jorge Martins\\
		86492 - Paulo Dias
		\end{flushleft}
	\end{minipage}
	~
	\begin{minipage}{0.4\textwidth}
		\begin{flushright}
		\large
		\textit{Docente}\\
		Tomás Grelha da Cunha
		\end{flushright}
	\end{minipage}
	
	\vfill
	\large\href{https://github.com/tecnico-distsys/A41-ForkExec}{\textbf{A41-ForkExec}}
	\vfill
	{\large\today}
	\vfill
\end{titlepage}

\begin{titlepage}

\tableofcontents
\end{titlepage}

\section{Definição do Modelo de Faltas}
Assume-se que:
\begin{itemize}
\item O sistema é assíncrono e a comunicação pode omitir mensagens
\begin{itemize}
\item Apesar do projeto usar HTTP como transporte, deve assumir-se que outros protocolos  de menor fiabilidade podem ser usados
\end{itemize}
\item Existem \textbf{N} gestores de réplicas e \textbf{N} é constante e igual a 3
\item Os gestores de réplicas podem falhar silenciosamente mas não arbitrariamente
\item No máximo, existe uma minoria de gestores de réplica em falha em simultâneo
\end{itemize}
\section{Solução de Tolerância a Faltas}
\begin{figure}[h!]
	\includegraphics[scale=.3]{../../domainsimple.jpeg}
	\caption{Simplificação do domínio da solução implementada}
	\label{fig:domain}
\end{figure}
\section{Descrição e breve explicação da solução}

Para manter a interface para o \textit{Hub} a	\textit{FrondEnd} do \textit{points} continua a implementar as funções necessárias para retornar os valores para o \textit{Hub}, no entanto esta classe apenas resolve a procura das replicas e reencaminha os pedidos para um \textit{PointsClient}.

No \textit{PointsClient} foi implementado o algoritmo \textit{QC} que será descrito com mais detalhe no final do relatório.
\section{Descrição de otimizações/simplificações}
   O algoritmo \textit{Quorum Consensus} suporta vários clientes.  E sempre que se pretende atualizar as replicas é necessário primeiro obter a \textit{tag} máxima de entre todas as replicas. No entanto como foi explicito no enunciado podemos assumir que apenas existe um cliente.
   
   Desta forma decidimos implementar uma cache no \textit{PointsClient} que guardava as informações de cada utilizador (\textit{tag} e \textit{points}), com o objetivo de deixar de fazer \textit{reads} antes de \textit{writes}, evitando-se o tempo de consultar as replicas, sendo apenas necessário ir consultar a cache. Assim as replicas continuam com a informação dos utilizadores correta, e a funcionar normalmente. Esta otimização funciona pois apenas existe um cliente e toda a informação passa por este.
   
   Por exemplo se existissem dois clientes esta otimização não funcionaria pois o cliente 1 poderia ter na \textit{cache} o valor X enquanto o cliente 2 escrevia nas replicas Y,   e quando fosse pedido o valor ao cliente 1 este não consultaria as replicas apenas a cache e devolveria e devolveria X, tornando-se um algoritmo inconsistente.
\section{Detalhe do protocolo (troca de mensagens)}
\subsection{Funções no Gestor de Réplica}
\subsubsection{\textit{read(String userEmail)}}
\begin{itemize}
\item Ao receber \textit{read(userEmail)}:
\begin{itemize}
\item[1.] Vai buscar a \textit{tag} e os \textit{pontos} associados ao \textit{userEmail}
\begin{itemize}
\item[1.1.] Se \textit{userEmail} não existe no sistema, então adiciona.
\end{itemize}
\item[2.] responde com \textbf{$Value=< pontos, tag >$} associado ao utilizador, em que: $tag=< seq, cid>$
\end{itemize}
\end{itemize}
\subsubsection{\textit{write(String userEmail, int points, Tag t)}}
\begin{itemize}
\item Ao receber \textit{write(userEmail,points, t)}:
\begin{itemize}
\item[1.] Vai buscar a \textit{tag} associada ao \textit{userEmail}
\begin{itemize}
\item[1.1.] Se \textit{userEmail} não existe no sistema, então adiciona.
\end{itemize}
\item[2.] Se \textit{t.getSeq() $>$ tag.getSeq()}:
\begin{itemize}
\item[2.1.] atualiza os \textit{pontos} do utilizador com \textit{points}
\item[2.2.] atualiza a \textit{tag} do utilizador com \textit{t}
\item[2.3.] responde \textit{ack}
\end{itemize}
\item[3.] Senão responde \textit{nack}
\end{itemize}
\end{itemize}
\subsection{Funções no \textit{Points Client}}
\subsubsection{Simplificação}
Para simplificar e reduzir a quantidade de texto repetido vamos definir o comportamento das seguintes funções:\\
\\
\textbf{getMaxValue(String userEmail)}
\begin{itemize}
\item Para cada replica:
\begin{itemize}
\item faz uma chamada assíncrona (\textit{readAsync(userEmail)}) para ir buscar o \textit{maxValue}
\end{itemize}
\item Enquanto o \textit{numRespostas} for menor que \textit{Q}.
\begin{itemize}
\item Para cada chamada
\begin{itemize}
\item Se já chegou, guarda o \textit{Value}
\item \textit{numRespostas++}
\end{itemize}
\end{itemize}
\item Determina e retorna \textit{maxValue}
\end{itemize}
\textbf{setMaxValue(String userEmail, Value value)}
\begin{itemize}
\item Para cada replica:
\begin{itemize}
\item executa \textit{writeAsync(userEmail, value.getVal(), value.getTag())}
\end{itemize}
\item Enquanto \textit{numRespostas} for menor que \textit{Q}
\begin{itemize}
\item Para cada chamada
\begin{itemize}
\item Se foi recebido \textit{ack}, \textit{numRespostas++}
\end{itemize}
\end{itemize}
\item Fim
\end{itemize}
\subsubsection{\textit{activateUser(String userEmail)}}
\begin{itemize}
\item maxValue = getMaxValue(userEmail)
\item Se a sequencia associada ao \textit{maxValue} é maior que 0 então é porque o utilizador existe e:
\begin{itemize}
\item \textit{throw EmailAlreadyExists}
\end{itemize}
\item Senão:
\begin{itemize}
\item \textit{setMaxValue(userEmail, newValue(maxValue.getVal(), maxValue.getTag()))}
\item \textit{addUserToCache(userEmail, maxValue)}
\end{itemize}
\end{itemize}
\subsubsection{\textit{pointsBalance(String userEmail)}}
\begin{itemize}
\item vai buscar o valor associado ao userEmail à cache
\begin{itemize}
\item senão existe:
\begin{itemize}
\item \textit{throw InvalidEmail}
\end{itemize}
\end{itemize}
\item \textit{return valueFromCache}
\end{itemize}
\subsubsection{\textit{addPoints(String userEmail, int pointsToAdd)}}
\begin{itemize}
\item \textit{value} = valor associado a \textit{userEmail} na cache
\item \textit{points} = \textit{value.getVal()} + \textit{pointsToAdd}
\item \textit{setMaxValue(userEmail, newValue(points, value.getTag()))}
\item atualizar a cache
\end{itemize}
\subsubsection{\textit{spendPoints(String userEmail, int pointsToSpend)}}
\begin{itemize}
\item \textit{value} = valor associado a \textit{userEmail} na cache
\item \textit{points} = \textit{value.getVal()} + \textit{pointsToAdd}
\item \textit{setMaxValue(userEmail, newValue(points, value.getTag()))}
\item atualizar a cache
\end{itemize}
\end{document}
