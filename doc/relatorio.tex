\documentclass[a4paper]{article}

\usepackage{hyperref}
\hypersetup{
    colorlinks=true,      
    urlcolor=blue,
}
\begin{document}
\begin{titlepage}
	\newcommand{\HRule}{\rule{\linewidth}{0.5mm}}
	\center
	\textsc{\LARGE INSTITUTO SUPERIOR TÉCNICO}\\[1.0cm]
	\textsc{\Large Sistemas Distribuídos}\\[0.5cm]
	\textsc{\large 3º Ano, 2º Semestre 2018/2019}\\[0.2cm]
	
	\HRule\\[0.4cm]
	{\huge\bfseries Relatório - Segunda Parte}\\[0.2cm]
	\HRule\\[1.5cm]
	
	\begin{minipage}{0.4\textwidth}
		\begin{flushleft}
		\large
		\textit{Autores}\\
		Filipe Marques\\
		Jorge Martins\\
		Paulo Dias
		\end{flushleft}
	\end{minipage}
	~
	\begin{minipage}{0.4\textwidth}
		\begin{flushright}
		\large
		\textit{Docente}\\
		Tomás Grelha da Cunha
		\end{flushright}
	\end{minipage}
	
	\vfill
	\large\href{https://github.com/tecnico-distsys/A41-ForkExec}{\textbf{A41-ForkExec}}
	\vfill
	{\large\today}
	\vfill
\end{titlepage}
\section{Definição do Modelo de Faltas}
Assume-se que:
\begin{itemize}
\item O sistema é assíncrono e a comunicação pode omitir mensagens
\begin{itemize}
\item Apesar do projeto usar HTTP como transporte, deve assumir-se que outros protocolos  de menor fiabilidade podem ser usados
\end{itemize}
\item Existem \textbf{N} gestores de réplicas e \textbf{N} é constante e igual a 3
\item Os gestores de réplicas podem falhar silenciosamente mas não arbitrariamente
\item No máximo, existe uma minoria de gestores de réplica em falha em simultâneo
\end{itemize}
\section{Figura da Solução de Tolerância a Faltas}
\section{Descrição da figura e breve explicação da solução}
\section{Descrição de otimizações/simplificações}
\section{Detalhe do protocolo (troca de mensagens)}
\begin{itemize}
\item 
\end{itemize}
\end{document}
